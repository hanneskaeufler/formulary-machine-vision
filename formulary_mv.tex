\documentclass[a4paper,12pt,pdftex]{scrreprt}

\title{Formelsammlung Machine Vision}
\author{Hannes Käufler}
\date{2013}

%
%	Neue Rechtschreibung und korrekte Silbentrennung
%
\usepackage[german, ngerman]{babel}

%
%	Umlaute korrekt darstellen
%
\usepackage[utf8]{inputenc}

%
%	Matrizen, Symbole ...
%
\usepackage{mathcomp}
\usepackage{amsmath}

\begin{document}
	\chapter{Introduction} % (fold)
	\label{cha:introduction}
	% chapter introduction (end)

	\chapter{Image Preprocessing} % (fold)
	\label{cha:image_preprocessing}
	
	\section{Convolution} % (fold)
	\label{sec:convolution}
	For two continuous functions f and g, the convolution is defined as\\
	\boxed{(f*g)(x)=\int_{-\infty}^{\infty}g(\tau)f(x-\tau)d\tau}
	being commutative, associative and linear. 
	% section convolution (end)

	\section{Fourier Transform} % (fold)
	\label{sec:fourier_transform}
	The Fourier transform of $f$ is given by:\\
	\boxed{\hat{f}(k):=\int_{-\infty}^{\infty}f(x)e^{-2\pi ikx}dx}
	With the inverese Fourier transform:\\
	\boxed{{f}(x):=\int_{-\infty}^{\infty}\hat{f}(k)e^{2\pi ikx}dk}
	% section fourier_transform (end)

	\section{Impulse Functions} % (fold)
	\label{sec:impulse_functions}
	Dirac impulse:
	\boxed{\delta(x)=
		\begin{cases}
			0 & if x \neq 0 \\
			\infty & if x = 0 \\
		\end{cases}}
	with \boxed{\int_{-\infty}^{\infty}\delta(x)dx=1}\\
	And a series of Dirac impulses:
	\boxed{\sum_{n=-\infty}^{\infty}\delta(x-nT)}
	% section impulse_functions (end)

	\section{Nyquist Sampling Theorem} % (fold)
	\label{sec:nyquist_sampling_theorem}
	If $f$ is band bounded with cutoff frequency $k_{0}$
	$\hat{f}(k)=0$ for all $k$ with $\lvert k \rvert \geq k_{0}$\\
	then it is completely determined by giving its ordinates at a series of points spaced at most
	$\frac{1}{2k_{0}}$\\ 
	meaning the sample frequency must be larger than $2k_{0}$
	% section nyquist_sampling_theorem (end)

	\section{Quantisation} % (fold)
	\label{sec:quantisation}
	Characteristic with equidistant steps of size $\Delta$\\
	\boxed{g(x)=max\{0,min\{g_{max},\left\lfloor \frac{I(x)}{\Delta}+\frac{1}{2} \right\rfloor\}\}}\\
	\boxed{h(x)=\Delta g(x)-\frac{\Delta}{2}}\\
	yielding an error of non-overdriven quantiser:\\
	\boxed{I(x)-h(x) \in{[-\frac{\Delta}{2},\frac{\Delta}{2}]} }
	% section quantisation (end)
	
	\section{Gamma Correction} % (fold)
	\label{sec:gamma_correction}
	\boxed{g_{out}=g_{max}\left( \frac{g_{in}}{g_{max}} \right)^{\gamma}} which keeps black and white, is a nonlinear transformation
	% section gamma_correction (end)

	\section{Models of Blur} % (fold)
	\label{sec:models_of_blur}
	Blur can be modeled with convolution\\
	\boxed{g_{blurred}=g_{sharp}*p} with $p$ modeling the blur\\
	\boxed{p_{motion}(x)=
		\begin{cases}
			\frac{1}{n} & if -n<x \leq 0 \\
			0 & otherwise \\
		\end{cases}} along x-axis by $n$ pixels\\
	\boxed{p_{Gauss}(x)=\frac{1}{\sqrt{2\pi \sigma^{2}}}e^{-\frac{1}{2}\frac{x^{2}}{\sigma^{2}}} }
	% section models_of_blur (end)

	\section{Wiener Deconvoution} % (fold)
	\label{sec:wiener_deconvoution}
	Restore a sharp image from a blurred image with a Wiener filter:
	\boxed{g_{blurred}=g_{sharp}*p+v} with $p$ being a point-spread function (blur), $v$ being pixel noise and assuming $g_{sharp}$ and $v$ being independent.\\
	With \boxed{g_{restored}=f*g_{blurred}}, find optimal $f$ that minimizes\\
	\boxed{e(k)=E\left[ \lvert \hat{g}_{sharp}(k)-\hat{g}_{restored}(k) \rvert^2 \right]} with $E$ being the expectation value.
	% section wiener_deconvoution (end)

	\section{Models of Noise} % (fold)
	\label{sec:models_of_noise}
	Statistical noise: \boxed{g_{noisy}(x,y)=g_{sharp}(x,y)+v(x,y)}\\
	Malfunctioning sensors: \boxed{g_{noisy}(x,y)=\begin{cases} g_{sharp}(x,y) & with \; probability \; p \\ arbitrary & otherwise \\ \end{cases}}
	% section models_of_noise (end)
	% chapter image_preprocessing (end)

	\chapter{Edge-Detection} % (fold)
	\label{cha:edge_detection}
	
	\section{Finding Edges} % (fold)
	\label{sec:finding_edges}
	Approximating a derivative by difference:
	\boxed{\frac{\partial g}{\partial x} \approx \frac{g(x+1)-g(x-1)}{2}}
	% section finding_edges (end)

	\section{Laplace Operator} % (fold)
	\label{sec:laplace_operator}
	The 2D analogon to the second order derivatie is the Laplace operator:\\
	\boxed{\bigtriangledown^{2}g=\frac{\partial^{2} g}{(\partial x)^{2}}+\frac{\partial^{2} g}{(\partial y)^{2}}} which can be appriximated with\\
	\boxed{\bigtriangledown^{2}g\approx g(x+1,y)+g(x-1,y)+g(x,y+1)+g(x,y-1)-4g(x,y)}\\
	Because the second order derivative is very noisy, so combine Laplacian with Gaussian smoothing:\\
	\boxed{\bigtriangledown^{2}(G*g)=(\bigtriangledown^{2}G)*g} which yields \\
	\boxed{\bigtriangledown^{2}G=\frac{x^{2}+y^{2}-2\sigma^{2}}{\sigma^{4}}G(x,y)} called Laplacian of Gaussian (LoG) or Mexican Hat\\
	The Laplacian of Gaussian can be approximated by calculating the Difference of Gaussian (DoG)\\
	\boxed{DoG(x,y)=G_{\sigma_{1}}(x,y)-G_{\sigma_{2}}(x,y)}
	% section laplace_operator (end)
	
	\section{Corner Detection} % (fold)
	\label{sec:corner_detection}
	Using dissimilarity measure to find patches of maximal dissimilarity for local moves:\\
	\boxed{d:= 
		\sum_{(u,v)\in rectangle}(g(u+\Delta u,v+\Delta v)-g(u,v))^{2} \approx \left ( \begin{array}{c}
			\Delta u \\ \Delta v \\
		\end{array} \right )^{T}
		\left ( \begin{array}{cc}
			\sum (\frac{\partial g}{u})^{2} & \sum \frac{\partial g}{u} \frac{\partial g}{v} \\ 
			\sum \frac{\partial g}{u} \frac{\partial g}{v} & \sum (\frac{\partial g}{v})^{2} \\
		\end{array} \right )
		\left ( \begin{array}{cc}
			\Delta u \\ \Delta v \\
		\end{array} \right )
	}\\
	For special choice of coordinate system it becomes a diagonal matrix:\\
	\boxed{d:=
		\left ( \begin{array}{c}
			\Delta u \\ \Delta v \\
		\end{array} \right )^{T}
		\left (
		\begin{array}{cc}
			\lambda_{1} & 0 \\
			0 & \lambda_{2}
		\end{array} \right )
		\left ( \begin{array}{cc}
			\Delta u \\ \Delta v \\
		\end{array} \right )
		= \lambda_{1}(\Delta u)^{2}+\lambda_{2}(\Delta v)^{2} \; \: w.l.o.g. \; \: \lambda_{1} \geq \lambda_{2} \geq 0
	}
	% section corner_detection (end)
	% chapter edge_detection (end)
	
	\chapter{Curve Fitting} % (fold)
	\label{cha:curve_fitting}
	
	\section{2D Geometry of Lines} % (fold)
	\label{sec:2d_geometry}
	Line in normal form: \boxed{0=\langle \vec{n},\vec{x} \rangle +c} so the distance of a point $\vec{r}$ from the line:\\
	\boxed{d=\| \vec{l}-\vec{r} \| = \lvert \langle \vec{n},\vec{r} \rangle +c \rvert}
	% section 2d_geometry (end)

	\section{Hough Transform} % (fold)
	\label{sec:hough_transform}
	Every line in 2D can be represented as: \boxed{x\:cos \phi + y\:sin \phi + c = 0} so there is a 2D space of all line represented by $(\phi,c)$ called Hough Space.
	% section hough_transform (end)

	\section{Total Least Squares} % (fold)
	\label{sec:total_least_squares}
	To estimate a line through a set of points we need to search the line parameters that minimise $d_{i}$\\
	\boxed{minimise \sum_{i=1}^{N} d_{i}^{2} \; subject \; to \; \langle \vec{n}, \vec{n} \rangle=1}
	Using the Lagrange function:\\
	\boxed{L(\vec{n},c,\lambda)=\sum_{i=1}^{n} d_{i}^{2} - \lambda(\langle \vec{n},\vec{n} \rangle -1)} then zeroing the partian derivative with respect to c and then to n, we can arrive at\\
	\boxed{
		\left ( \begin{array}{cc}
			\alpha & \beta \\
			\beta & \gamma \\
		\end{array}	\right )
		\vec{n} = \lambda \vec{n}
	} hence $\lambda$ is Eigenvalue and $\vec{n}$ is Eigenvector. We get two solutions of Eigenvalue Problem: \boxed{\lambda_{1} \geq \lambda_{2} \geq 0} where $\lambda_{2}$ minimises distances and $\lambda_{1}$ maximises distances
	% section total_least_squares (end)

	\section{Weighted Least Squares} % (fold)
	\label{sec:weighted_least_squares}
	Because of outliers introduce weights $w_i \geq 0$, one for each edge point, then solve:\\
	\boxed{minimise \sum_{i=1}^{N} w_{i} d_{i}^{2} \; subject \; to \; \langle \vec{n}, \vec{n} \rangle=1}.
	But how to choose $w_{i}$? Replace $d_{i}$ by a term that grows more slowly!
	% section weighted_least_squares (end)

	\section{M-Estimators} % (fold)
	\label{sec:m_estimators}
	M-Estimator means 
	\boxed{minimise \sum_{i=1}^{N} \rho(d_{i}) \; subject \; to \; \langle \vec{n}, \vec{n} \rangle=1} with a suitable function $\rho$\\
	The derivatives of the Lagrange function of M-Estimators approach and Weighted Least Squares approach are equal if \boxed{w_{i}=\frac{\frac{\partial \rho(d_{i})}{\partial d_{i}}}{2d_{i}}}. So running Weighted Least Squares with appropriate weights implements M-Estimators. E.g.:\\
	\boxed{\rho_{Cauchy}: d \mapsto c^{2} log(1+\frac{d^{2}}{c^{2}})}\\
	\boxed{\rho_{Huber}: d \mapsto 
		\begin{cases}
			d^{2} & if \lvert d \rvert \leq k \\
			2k \lvert d \rvert - k^{2} & otherwise \\
		\end{cases}}
	% section m_estimators (end)

	\section{Least Trimmed Sum of Squares Estimator (LTS)} % (fold)
	\label{sec:least_trimmed_sum_of_squares_estimator_}
	\boxed{minimise \sum_{i=1}^{p} d_{i:N} \; subject \; to \; \langle \vec{n}, \vec{n} \rangle=1} with $p<N$ and $d_{i:N}$ the i-th element in the ordered list of point-distances. $p$ is typically given as a percentage of N.
	% section least_trimmed_sum_of_squares_estimator_ (end)

	\section{RANSAC} % (fold)
	\label{sec:ransac}
	Search a line that passes nearby as many points as possible.\\
	\boxed{minimise \sum_{i=1}^{N} \sigma(d_{i}) \; with \; \sigma(d_{i})= \begin{cases}
		0 & if \lvert d_{i} \rvert \leq \theta \\
		1 & if \lvert d_{i} \rvert > \theta\\
	\end{cases}}
	% section ransac (end)

	\section{Estimating Circles} % (fold)
	\label{sec:estimating_circles}
	Parametric representation of circles:\\
	\boxed{(x-m_{1})^{2}+(y-m_{2})^{2}-r^{2}=0}\\
	Euclidean distance of point (x,y) from the circle:\\
	\boxed{d_{E}=\left \lvert \sqrt{(x-m_{1})^{2}+(y-m_{2})^{2}}-r \right \rvert}\\
	Algebraic distance:\\
	\boxed{d_{A}=\left \lvert (x-m_{1})^{2}+(y-m_{2})^{2}-r^{2} \right \rvert}
	Because minimising the Euclidean distance cannot be solved analytically, minimalize the sum of algebraic distances. Use a weighted approach, chose $w_{i}$ so that we implement Euclidien by using weighted Algebraic.
	% section estimating_circles (end)
	% chapter curve_fitting (end)

	\chapter{Color} % (fold)
	\label{cha:color}
	% chapter color (end)

	\chapter{Segmentation} % (fold)
	\label{cha:segmentation}
	
	\section{Level Set Evolution} % (fold)
	\label{sec:level_set_evolution}
	Two class segmentation can be represented by indicatior function:\\
	\boxed{\phi(\vec{x}) \begin{cases}
		< 0 & if\:pixel\:\vec{x}\:belongs\:to\:segment\\
		> 0 & if\:pixel\:\vec{x}\:belongs\:to\:background\\
	\end{cases}} with \boxed{\lvert \phi(\vec{x}) \rvert = distance\:of\:\vec{x}\:from\:contour}\\
	Modeling temporal evolution of signed distance function with \boxed{\phi(\vec{x},t)}. When tracking a point \boxed{\vec{x}(t)} on the boundary over time, it obviously follows \boxed{\phi(\vec{x}(t),t)=0\:for\:all\:t}\\
	\boxed{0=\frac{d\phi(\vec{x}(t),t)}{dt}=\bigtriangledown \phi \frac{\partial \vec{x}}{\partial t}+\frac{\partial \phi}{\partial t}}. Solving this for $\frac{\partial \phi}{\partial t}$ yields:\\
	\boxed{\frac{\partial \phi}{\partial t}=-\bigtriangledown \phi \frac{\partial \vec{x}}{\partial t}}
	% section level_set_evolution (end)

	\section{Mumford-Shah} % (fold)
	\label{sec:mumford_shah}
	Shrink contour if \boxed{(I-C_{in})^{2} > (I-C_{out})^{2}}\\
	Expand contour if \boxed{(I-C_{in})^{2} < (I-C_{out})^{2}}\\
	Mumford-Shah based segmentation:\\
	\boxed{\frac{\partial \vec{x}}{\partial t}=\frac{\bigtriangledown \phi}{\| \bigtriangledown \phi \|} (C+P-\lambda_{1}(I-C_{in})^{2}+\lambda_{2}(I-C_{out})^{2})}
	% section mumford_shah (end)
	% chapter segmentation (end)

	\chapter{Camera Optics} % (fold)
	\label{cha:camera_optics}

	\section{World To Image Mapping} % (fold)
	\label{sec:world_to_image_mapping}
	
	Point $(x,y,z)$ is projected onto $(x',y')$ via intercept theorem:\\
	\boxed{z\left(
		\begin{array}{c}
			x'\\
			y'\\
		\end{array}
		\right) = \left(
		\begin{array}{cc}
			f & 0\\
			0 & f\\	
		\end{array} \right) \left(
		\begin{array}{c}
			x\\
			y\\
		\end{array} \right)
	}\\
	Mapping from camera to image frame:\\
	\boxed{
		\left(
		\begin{array}{c}
			u\\
			v\\
		\end{array}
		\right) = \left(
		\begin{array}{cc}
			\alpha & -\beta cot\theta\\
			0 & \frac{\beta}{sin\theta}\\	
		\end{array} \right) \left(
		\begin{array}{c}
			x'\\
			y'\\
		\end{array} \right) + \left(
		\begin{array}{c}
			u_{0}\\
			v_{0}\\
		\end{array} \right)
	}\\
	Mapping from world to camera frame:\\
	\boxed{
		\left(
		\begin{array}{c}
			x\\
			y\\
			z\\
		\end{array} \right)=
		R \left(
		\begin{array}{c}
			\xi\\
			\eta\\
			\zeta\\
		\end{array} \right) + \vec{t}
	}\\
	Rewriting all those, given $(\xi,\eta,\zeta)$ we can calculate $(u,v)$ by\\
	\boxed{
		\left(
		\begin{array}{c}
			\tilde{x}\\
			\tilde{y}\\
			\tilde{z}\\
		\end{array} \right)= A \left ( R \mid \vec{t} \right) \left(
		\begin{array}{c}
			\xi\\
			\eta\\
			\zeta\\
			1\\
		\end{array} \right)
	} and 
	\boxed{
		\left(
		\begin{array}{c}
			u\\
			v\\
		\end{array} \right) = \frac{1}{\tilde{z}} \left(
		\begin{array}{c}
			\tilde{x}\\
			\tilde{y}\\
		\end{array} \right)
	}\\
	The other way round, we can obtain $(\xi,\eta,\zeta)$ given $(u,v)$ by\\
	\boxed{
		\left(
		\begin{array}{c}
			\xi\\
			\eta\\
			\zeta\\
		\end{array} \right)= z R^{T} A^{-1} \left(
		\begin{array}{c}
			u\\
			v\\
			1\\
		\end{array} \right) - R^{T}\vec{t}\;\;\;with\:z\leq 0
	}
	% section world_to_image_mapping (end)

	\section{Snell´s law} % (fold)
	\label{sec:snell_s_law}
	\boxed{n_{e}sin\theta_{e}=n_{t}sin\theta_{t}} with \boxed{n_{medium}=\frac{v_{vacuum}}{v_{medium}}}
	% section snell_s_law (end)

	\section{Thin lenses} % (fold)
	\label{sec:thin_lenses}
	Via Intercept theorem, we get the lens equation:\\
	\boxed{\frac{1}{f}=\frac{1}{b}+\frac{1}{g}}
	% section thin_lenses (end)

	\section{Depth of Field} % (fold)
	\label{sec:depth_of_field}
	Via intercept theorem with:\\
	\boxed{\bigtriangledown g=g_{far}-g_{near}=2\frac{g d_{h}(g-f)}{d_{h}^{2}-(g-f)^{2}}} with \boxed{d_{h}=\frac{Df}{\epsilon}} as hyperfocal distance.
	% section depth_of_field (end)

	\section{Radial Distortion} % (fold)
	\label{sec:radial_distortion}
	Mathematical modeling with even polynominals:\\
	\boxed{\left(
		\begin{array}{c}
			x_{d}\\
			y_{d}\\
		\end{array} \right)=(1+k_{1}r^{2}+k_{2}r^{4})\left(
		\begin{array}{c}
			x'\\
			y'\\
		\end{array} \right)} with $r^{2}=(x')^{2}+(y')^{2}$
	% section radial_distortion (end)

	\section{Camera Calibration} % (fold)
	\label{sec:camera_calibration}
	Tsai´s approach:\\
	From world-to-image-mapping take $M:=A\left(R\mid\vec{t}\right)$ in which M is a 3x4 matrix.\\
	Determine camera parameters by minimizing sum of squares, which means zeroing partial derivaties.\\
	% section camera_calibration (end)

	\section{Telecentric Lenses} % (fold)
	\label{sec:telecentric_lenses}
	Maginification of telecentric lens: \boxed{B=\frac{b-f}{f}G}\\
	Depth of field:\boxed{\bigtriangledown g=2\frac{\epsilon}{D}(g-f)}
	% section telecentric_lenses (end)
	% chapter camera_optics (end)

	\chapter{Illumination} % (fold)
	\label{cha:illumination}
	Irradiance of a surface point, meaning amount of incident light:\\
	\boxed{I(\theta_{i},\phi_{i})}\\
	Radiance of a surface point, meaning amount of emitted light:\\
	\boxed{I(\theta_{e},\phi_{e})}\\
	Total irradiance of a surface patch (in sperical coordinates):\\
	\boxed{I_{0}=\int_{0}^{2\pi}\int_{0}^{\frac{\pi}{2}}I(\theta_{i},\phi_{i})sin\theta_{i}cos\theta_{i}d\theta_{i}d\phi_{i}}\\
	Reflectance of a surface patch can be modeled by the bidirectional reflectance distribution function (BRDF):\\
	\boxed{f(\theta_{e},\phi_{e}\mid\theta_{i},\phi_{i})} describing how much light is radiated in direction $(\theta_{e},\phi_{e})$ if one unit of light is irradiated from direction $(\theta_{i},\phi_{i})$

	\section{Reflectance} % (fold)
	\label{sec:reflectance}
	So total amount of radiated light:\\
	\boxed{I_{0}=\int_{0}^{2\pi}\int_{0}^{\frac{\pi}{2}}f(\theta_{e},\phi_{e}\mid\theta_{i},\phi_{i})I(\theta_{i},\phi_{i})sin\theta_{i}cos\theta_{i}d\theta_{i}d\phi_{i}}\\
	Lambertian Reflectance:\\
	\boxed{f(\theta_{e},\phi_{e}\mid\theta_{i},\phi_{i})=\frac{1}{\pi}} hence: \boxed{L(\theta_{e},\phi_{e})=\frac{1}{\pi}L_{0}}\\
	Specular Reflectance, mirroring effect:\\
	\boxed{f(\theta_{e},\phi_{e}\mid\theta_{i},\phi_{i})=\frac{\delta(\theta_{e}-\theta_{i})\delta(\phi_{e}-\phi_{i}-\pi)}{sin\theta_{i}cos\theta_{i}}} hence:
	\boxed{L(\theta_{e},\phi_{e})=I(\theta_{e},\phi_{e}-\pi)}
	Combinations of Reflectance:\\
	\boxed{f(\theta_{e},\phi_{e}\mid\theta_{i},\phi_{i})=\frac{\eta}{\pi}+(1-\eta)\frac{\delta(\theta_{e}-\theta_{i})\delta(\phi_{e}-\phi_{i}-\pi)}{sin\theta_{i}cos\theta_{i}}}
	% section reflectance (end)
	% chapter illumination (end)

	\chapter{3D Reconstruction} % (fold)
	\label{cha:3d_reconstruction}

	\section{Stereovision by Triangulation} % (fold)
	\label{sec:stereovision_by_triangulation}
	For camera 1:\\
	\boxed{
		\left(
		\begin{array}{c}
			\xi\\
			\eta\\
			\zeta\\
		\end{array} \right)=
		z R^{T} A^{-1} \left(
		\begin{array}{c}
			u\\
			v\\
			1\\
		\end{array} \right) - R^{T} \vec{t}
	} with unknown depth $z$\\
	Same for camera 2 and equaling:\\
	\boxed{z\vec{p}-\vec{q}=z'\vec{p}'-\vec{q}'} is overdetermined, lines of sight might be skewed. Minimize square of norm instead!\\
	% section stereovision_by_triangulation (end)

	\section{Phase Shift} % (fold)
	\label{sec:phase_shift}
	Measure gray value of pixel 4 times with phase shift multiples of $\frac{\pi}{2}$:\\
	\boxed{g_{i}=A\;sin\left( \frac{u}{w}2\pi + \dotsb \right)+B} with wavelength $w$. Solve for A and B.\\
	\boxed{u=\frac{atan2(g_{1}-g_{3},g_{2}-g_{4})}{2\pi}w+kw} only if $A\neq0$
	% section phase_shift (end)
	% chapter 3d_reconstruction (end)

	\chapter{Pattern Recognition} % (fold)
	\label{cha:pattern_recognition}

	\section{Bayesian Classification} % (fold)
	\label{sec:bayesian_classification}
	assign new object to the class with the largest posteriori probability:\\
	\boxed{\hat{c}=arg\:max\:P(c \mid x_{new})} and \boxed{P(c \mid x) \propto P(x \mid c)P(c)}
	% section bayesian_classification (end)

	\section{Linear Classification} % (fold)
	\label{sec:linear_classification}
	A linear classifier is a function that implements a function of the kind:\\
	\boxed{\vec{x}\mapsto \begin{cases}
		+1 & if \langle \vec{x},\vec{w} \rangle + b \geq 0\\
		-1 & otherwise
	\end{cases}} with $\vec{w}$ the weight vector and $b$ the bias weight.\\
	Margin, meaning the minimal distance between a hyperplane and the convex hull of the training patterns:\\
	\boxed{\rho=min \left( d^{(i)}\frac{\langle \vec{x}^{(i)},\vec{w} \rangle + b}{\| \vec{w} \|} \right)}\\
	% section linear_classification (end)
	
	\section{Support Vector Machine} % (fold)
	\label{sec:support_vector_machine}
	SVM is a linear classifier that maximizes the margin $\rho$. Training a SVM means solving:\\
	\boxed{maximize\quad\rho^{2}\quad subject\;to\quad d^{(i)}\frac{\langle \vec{x}^{(i)},\vec{w} \rangle +b}{\| \vec{w} \|}\geq \rho \qquad for\;all\;i}\\
	Where $\| \vec{w} \|$ is a degree of freedom, so set $\| \vec{w} \| = \frac{1}{\rho}$ for convenience. That leaves us to:\\
	\boxed{minimize\quad \frac{1}{2} \| \vec{w} \|^{2}\quad subject\;to\quad d^{(i)} (\langle \vec{x}^{(i)},\vec{w} \rangle +b)\geq 1 \qquad for\:all\: i}
	% section support_vector_machine (end)

	\section{Fault-tolerant SVM} % (fold)
	\label{sec:fault_tolerant_svm}
	Soft margin SVM:\\
	\boxed{minimize\quad \frac{1}{2} \| \vec{w} \|^{2} + C\sum_{i} \xi_{i}\quad subject\;to\quad d^{(i)} (\langle \vec{x}^{(i)},\vec{w} \rangle +b)\geq 1-\xi_{i} \qquad for\:all\: i}
	% section fault_tolerant_svm (end)

	\section{Nonlinear SVM} % (fold)
	\label{sec:nonlinear_svm}
	Assume nonlinear transformation:\\
	\boxed{\theta : \begin{cases}
		R^{n} \to R^{m}\\
		\vec{x} \mapsto \theta(\vec{x}) = \vec{X}\\
	\end{cases}}\\
	Useful kerner-functions are:\\
	\boxed{K(\vec{x},\vec{y}) = \langle \vec{x},\vec{y} \rangle} dot product\\
	\boxed{K(\vec{x},\vec{y}) = (\langle \vec{x}, \vec{y} \rangle)^{d}\quad or \quad (\langle \vec{x}, \vec{y} \rangle +1)^{d}} polynominal kernels\\
	\boxed{K(\vec{x},\vec{y}) = e^{-\frac{\| \vec{x} - \vec{y} \|^{2}}{2\sigma^{2}}}}
	% section nonlinear_svm (end)

	\section{Validation Process} % (fold)
	\label{sec:validation_process}
	Expected risk of misclassification: risk of false negative + risk of false positiv\\
	\boxed{E=\int_{A-}P(+)p_{+}(x)dx + \int_{A+}P(-)p_{-}(x)dx}\\
	Which is unknown but can be approximated by\\
	\boxed{E \approx \frac{n_{fn}+n_{fp}}{n}}, where $n$ is the number of elements in the sample set, $n_{fp}$ is the number of false positives in the sample set, $n_{fn}$ is the number of false negatives in the sample set.
	% section validation_process (end)

	\section{Haar Features} % (fold)
	\label{sec:haar_features}
	Calculating Haar features the naive way:\\
	\boxed{s=\sum_{u=u_{0}}^{u_{0}+w-1} \sum_{v=v_{0}}^{v_{0}+h-1} g(u,v)}\\
	The integral image:\boxed{I(x,y):=\sum_{u=0}^{x} \sum_{v_0}^{y} g(u,v)} makes calculating s simply a 4-way lookup on I:\\
	\boxed{s=I(u_{0}+w-1,v_{0}+h-1)-I(u_{0}-1,v_{0}+h-1)+I(u_{0}-1,v_{0}-1)-I(u_{0}+w-1,v_{0}-1)}
	% section haar_features (end)

	\section{Ensembles} % (fold)
	\label{sec:ensembles}
	Sum up all predictions and compare with zero:\\
	\boxed{ensemble(\vec{x})=sign\left( \sum_{j=1}^{k} c_{i}(\vec{x}) \right)}
	% section ensembles (end)

	\section{Boosting} % (fold)
	\label{sec:boosting}
	Introduce weight $\gamma_{i}$ for each training pattern, applied to e.g. soft margin SVM this yields:\\
	\boxed{minimize\quad \frac{1}{2} \| \vec{w} \|^{2} + C\sum_{i} \gamma_{i} \xi_{i}\quad subject\;to\quad d^{(i)} (\langle \vec{x}^{(i)},\vec{w} \rangle +b)\geq 1-\xi_{i} \qquad for\:all\: i}
	% section boosting (end)

	\section{AdaBoost} % (fold)
	\label{sec:adaboost}
	Is an implementation of Boosting. With the training error bounded to\\
	\boxed{\Pi_{t=1}^{T}\left( 2\sqrt{\epsilon_{t}(1-\epsilon_{t})}\right) \leq exp \left\{ -2 \sum_{t=1}^{T} (\frac{1}{2}-\epsilon_{t})^{2} \right\}}\\
	If all $\epsilon_{t} < \frac{1}{2}$ and $T \to \infty$ AdaBoost yields a perfect classifier.
	% section adaboost (end)

	% chapter pattern_recognition (end)

\end{document}